\documentclass[cjk]{beamer}  % 使用Beamer包
%\hypersetup{pdfpagemode=FullScreen}
\usepackage{CJK}   % 中文环境
\usepackage{indentfirst}
%\usepackage{beamerthemesplit}

\usepackage{makeidx}
%\usepackage[dvipdfm,CJKbookmarks,bookmarks=true,bookmarksnumbered=true]{hyperref}
%\usepackage[CJKbookmarks,bookmarks=true,bookmarksnumbered=true]{hyperref}
\usepackage{listings}
\usepackage{verbatim}

\hypersetup{colorlinks, linkcolor=blue, citecolor=blue,
    urlcolor=blue,
    plainpages=flase,
    pdfcreator=tex,
    bookmarksopen=true,
    pdfhighlight=/P,
    pdfauthor={Yao Qi <qiyaoltc@gmail.com>},
    pdfcreator={cTeX},
%    pdftitle={iptables 入门},
%    pdfkeywords={iptables 入门 Rae szlug},
    pdfstartview=FitH,
    pdfpagemode=UseOutlines,%UseOutlines, %None, FullScreen, UseThumbs
}

%\usepackage[all,bottom]{draftcopy}
%\draftcopyName{Copyright by Rae <crquan@gmail.com>, 2006}{50}
%\draftcopySetGrey{0.8} \draftcopyPageTransform{55 rotate}
%\draftcopyPageX{80}\draftcopyPageY{-25}

\lstset{
  basicstyle=\tiny,          % print whole listing small
  showstringspaces=false,
  numbers=left,
  numberstyle=\tiny,
  % labelstep=1,
  % commentstyle=\textsl,
  keywordstyle=\color{green}\bfseries\underbar,     % underlined bold red keywords
  identifierstyle={},         % nothing happens to other identifiers
  commentstyle=\mdseries\color{blue}, % white comments
  stringstyle=\color{red}\ttfamily}      % typewriter type for strings
% stringspaces=false}         % no special string spaces



%\usetheme{Madrid}  % 采用的主题
\usetheme{Warsaw}
%\usetheme{Rochester}
%\usetheme{Berkeley}
%\usetheme{PaloAlto}
%\usetheme{Montpellier}
%\usetheme{Berlin}
%\usetheme{Ilmenau}
%\usetheme{Dresden}

%\usecolortheme{albatross} % 采用的配色。
%\usecolortheme{beaver} white background
%\usecolortheme{beetle}
%\usecolortheme{dolphin}
%\usecolortheme{fly}
%\usecolortheme{lily}

%\useoutertheme{tree}

\usefonttheme{structureitalicserif}

\begin{document}
\begin{CJK}{UTF8}{gbsn}

\title{Displaced stepping in GDB and its implementation in Thumb-2}
\author{\href{mailto:qiyaoltc@gmail.com}{齐尧}}
\date{HelloGCC 2011 \\ \today}

% 生成上面定义的"\title", "\author", "\date"等信息
\frame{\titlepage}


% 生成目录菜单
\section{目录及流程}
\frame{
    \tableofcontents
}

\section{关于我的话题}

\begin{frame}
  \frametitle{在这四十分钟里边}
  \begin{columns}
    \column{5cm}
    \begin{overprint}

      \begin{block}{你能知道的}
        \begin{itemize}
        \item 我叫齐尧 \href{mailto:qiyaoltc@gmail.com}{qiyaoltc@gmail.com}
        \item 什么是displaced stepping
        \item 如何为一些新的指令支持displaced stepping
        \item 将来的多核或者多线程调试可能是什么样子
        \end{itemize}
      \end{block}
    \end{overprint}

    \column{5cm}
    \begin{overprint}
      \begin{block}{你不可能知道的}
        \begin{itemize}
        \item GDB里边的displaced stepping 的代码到底是怎样做的
        \item GDB里边的displaced stepping 为什么设计成这个样子
        \item 为了多核或者多线程调试,具体应该做什么
        \end{itemize}
      \end{block}
    \end{overprint}
  \end{columns}
\end{frame}

\section{ Displaced stepping }

\subsection{什么是displaced stepping?}
\frame{
  \frametitle{什么是in-place stepping?}

\begin{figure}
  \includegraphics{../../book/breakpoint/break.4}\hfill
  \includegraphics{../../book/breakpoint/break.5}\hfill
  \includegraphics{../../book/breakpoint/break.6}\hfill
  \includegraphics{../../book/breakpoint/break.2}
  \caption{in-place stepping}
  \label{fig:inplace}
\end{figure}

in-place stepping 的问题 (我在HellGCC 2010讲过了)
  \begin{itemize}
  \item GDB 和其它调试器的缺省和基本方式
  \item 不适合调试多线程程序
  \item 对程序的干扰比较大
  \end{itemize}
}

\frame{
  \frametitle{什么是displaced stepping?}

\begin{figure}
  \includegraphics[height=2.5cm]{../../book/breakpoint/break.4}\hfill
  \includegraphics[height=3cm]{../../book/breakpoint/break.7}\hfill
  \includegraphics[height=2.5cm]{../../book/breakpoint/break.8}
  \caption{displaced stepping}
  \label{fig:inplace}
\end{figure}

displaced stepping 的考虑
  \begin{itemize}
  \item 把和inst.b\textbf{等价}的指令inst.b1,inst.b2拷贝到\texttt{scratch pad},然后执行。
  \item 如何保证 inst.b == inst.b1 + inst.b2?
  \item \texttt{scratch pad}在哪里?
  \end{itemize}

}
\subsection{Displaced stepping的设计考虑}
\frame{
  \frametitle{Displaced stepping的设计考虑}

  \begin{columns}
    \column{3cm}
    \begin{overprint}
      \begin{block}{}
        \begin{figure}
          \includegraphics[height=4cm]{../../book/breakpoint/break.9}
        \end{figure}
      \end{block}
    \end{overprint}

    \column{7cm}
    \begin{overprint}
      \begin{block}{设计考虑}
        \begin{itemize}
          \item 在内存空间中找到一段,可以执行代码,同时不能影响到程序其他部分的执行;\texttt{\_start};
          \item 对于任意一条指令inst.b,找到若干条与其在语义上完全一致的指令;和位置无关的指令是可以直接替代的,比如\texttt{mov r0, r1};
        \end{itemize}
      \end{block}
    \end{overprint}

  \end{columns}
}
    
\section{Displaced stepping在GDB中的实现}

\subsection{Displaced stepping in GDB}
\frame{
  \frametitle{Displaced stepping in GDB}

  \begin{overprint}

    \begin{block}{过程}
      \begin{itemize}
      \item 确定scratchpad的位置和长度 {\color{blue}\texttt{gdbarch\_max\_insn\_length}} {\color{blue}\texttt{gdbarch\_displaced\_step\_location}}
      \item 保存scratch pad内容,准备把需要的指令拷贝过去
      \item 分析当前指令,生成需要拷贝到scratch pad的指令,并且保存需要的寄存器  {\color{blue}\texttt{gdbarch\_displaced\_step\_copy\_insn}}
      \item 拷贝指令到scratch pad,加上断点指令。设置pc,让程序继续执行
      \item 命中断点后,恢复scratch pad,进行cleanup,恢复寄存器等等。{\color{blue}\texttt{gdbarch\_displaced\_step\_fixup}}
      \end{itemize}
    \end{block}

  \end{overprint}

}

\subsection{Displaced stepping for Thumb-2}
\begin{frame}[fragile]
  \frametitle{Displaced stepping for Thumb-2}

  \begin{itemize}
  \item 指令解码
  \item 选择替代指令
  \item 选择合适的cleanup操作
  \end{itemize}

  \begin{columns}
    \column{7cm}
    \begin{overprint}
      \begin{block}{\texttt{LDR Rd \#imm8}}
        \begin{itemize}
        \item 选择替代指令 \texttt{LDR R0, [R2, R3]}
        \item 保存\texttt{R0}, \texttt{R2}, 和\texttt{R3},
        \item 在\texttt{R2}写入PC,在\texttt{R3}中写入\texttt{\#imm8}
        \item 在scratchpad执行 \texttt{LDR R0, [R2, R3]}
        \item 把\texttt{R0}的内容读出,写到\texttt{Rd},恢复\texttt{R0}, \texttt{R2}, 和\texttt{R3}。
        \end{itemize}

      \end{block}
    \end{overprint}

    \column{4.5cm}
    \begin{overprint}
      \begin{block}{\texttt{BL \#imm}}
        \begin{itemize}
        \item 选在替代指令 \texttt{NOP}
        \item 计算目标地址\texttt{DEST}
        \item 在scratchpad执行 \texttt{NOP}
        \item 把\texttt{PC}的内容设置为\texttt{DEST},同时更新\texttt{LR}
        \end{itemize}
      \end{block}
    \end{overprint}
  \end{columns} 
\end{frame}

\frame{
  \frametitle{Displaced stepping for Thumb-2}
    \begin{columns}
    \column{5cm}
    \begin{overprint}
      \begin{block}{实现工作}
        \begin{itemize}
        \item 把每一条\texttt{Thumb}指令,做上一页的分析和操作,用代码实现。
        \item 代码量在两千行
        \item 从\texttt{2010-12} 到 \texttt{2011-09}
        \end{itemize}
      \end{block}
    \end{overprint}

    \column{6.5cm}
    \begin{overprint}
      \begin{block}{总结经验}
        \begin{itemize}
        \item 需要十分熟悉处理器的指令
        \item 需要及其细心地阅读指令手册
        \item 和社区打交到的经验
        \end{itemize}
      \end{block}
    \end{overprint}
  \end{columns} 
}

\section{Displaced stepping 的用途}
\subsection{Non-stop}
\frame{
  \frametitle{Non-stop}
}

\section{多线程调试或者多核处理器调试展望}
\frame{
  \frametitle{多线程调试或者多核处理器调试展望}

  \begin{columns}
    \column{5cm}
    \begin{overprint}
      \begin{block}{传统调试 vs 多核多线程}
        \begin{itemize}
        \item 传统调试基于这样的假设:
          \begin{itemize}
          \item 错误可重现
          \item 调试对程序无干扰或者少干扰
          \end{itemize}
        \item 多核多线程程序:
          \begin{itemize}
          \item 不确定性,问题也许不能重现
          \item 常规的测试没有意义,或者意义减少
          \end{itemize}
        \end{itemize}
      \end{block}
    \end{overprint}

    \column{6.5cm}
    \begin{overprint}
    \begin{block}{调试手段的变化(个人愚见)}
      \begin{itemize}
      \item 更多的tracing, 更少的peeking。采用高效的trace,采集数据。
      \item 更多的有算法的分析,更少的手工检查。利用先进的算法分析采集到的数据,推倒出程序中的问题。
      \item 更多的\textbf{工具引导人},更少的\textbf{人使用工具}。由工具重现问题的环境,然后再有人去检查程序的问题。
      \item visualization和重新定义\textbf{程序性能}和profiling。
      \end{itemize}
    \end{block}

    \end{overprint}
  \end{columns}
}

\frame{
  \frametitle{The End}

    \begin{overprint}

    \begin{block}{\texttt{Q \& A}}
    \end{block}

  \end{overprint}
}

\end{CJK}
\end{document}
